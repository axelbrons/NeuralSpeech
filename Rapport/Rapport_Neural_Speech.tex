\documentclass[a4paper,11pt]{article}

\usepackage[utf8]{inputenc} % allow utf-8 input
\usepackage[T1]{fontenc}    % use 8-bit T1 fonts
\usepackage{hyperref}       % hyperlinks
\usepackage{url}            % simple URL typesetting
\usepackage{booktabs}       % professional-quality tables
\usepackage{amsfonts}       % blackboard math symbols
%\usepackage{nicefrac}       % compact symbols for 1/2, etc.
\usepackage{microtype}      % microtypography

\usepackage{amssymb,amsmath}
%\usepackage{graphicx}
%\usepackage{babel}
%\usepackage[veryoldstyle,largesmallcaps]{kpfonts}
\usepackage{lmodern}
\usepackage{parskip}
\usepackage{capt-of}
\usepackage{xfp}
 \usepackage{xcolor}
\usepackage{tikz}
\usepackage{pgfplots}
\usepackage[ruled, french]{algorithm2e}
\usepackage{pgfplotstable}
\pgfplotsset{compat = newest}
\usepackage[french,english]{babel}
\usepackage{float}
%\graphicspath{ {./images/} }
\usepackage{mathtools}
\usepackage[margin=1in]{geometry} % Ajuste les marges
\usepackage{lipsum} % Pour générer du texte factice
\usepackage{geometry}
\geometry{left=2cm,right=2cm,bottom=3cm, top=3cm}
\usepackage{fancyhdr}
\pagestyle{fancy}

\renewcommand{\headrulewidth}{0pt}
\fancyhead[L]{\includegraphics[width=2cm]{images/ece.png}}
\fancyhead[R]{ING{3} Groupe{3}}
\definecolor{ece}{RGB}{0, 122, 123}

\usepackage{listings}
\usepackage{xcolor}

\definecolor{codegreen}{rgb}{0,0.6,0}
\definecolor{codegray}{rgb}{0.5,0.5,0.5}
\definecolor{codepurple}{rgb}{0.58,0,0.82}
\definecolor{backcolour}{rgb}{0.95,0.95,0.92}

\lstdefinestyle{mystyle}{
    backgroundcolor=\color{backcolour},   
    commentstyle=\color{codegreen},
    keywordstyle=\color{magenta},
    numberstyle=\tiny\color{codegray},
    stringstyle=\color{codepurple},
    basicstyle=\ttfamily\footnotesize,
    breakatwhitespace=false,         
    breaklines=true,                 
    captionpos=b,                    
    keepspaces=true,                 
    numbers=left,                    
    numbersep=5pt,                  
    showspaces=false,                
    showstringspaces=false,
    showtabs=false,                  
    tabsize=2
}

\lstset{style=mystyle}


\begin{document}

\begin{titlepage}
    \noindent
    \begin{center}
        \centering
        \includegraphics[width=7cm]{images/ece.png} % Remplacez par le chemin de votre logo
    \end{center}
	\vfill
	
	\vspace{1cm}
	\makebox[\linewidth]{\rule{\textwidth}{0.4pt}} % Ligne horizontale supérieure
    
    \centering
    {\Huge Rapport Neural Speech\par}
    
    \makebox[\linewidth]{\rule{\textwidth}{0.4pt}} % Ligne horizontale inférieure
    \vspace{1cm}	
	
    \begin{minipage}[t]{0.45\textwidth}
        \centering
        {\Large Axel Bröns \\ \texttt{\href{mailto:axel.brons@edu.ece.fr}{axel.brons@edu.ece.fr}}}
    \end{minipage}
    \hfill
    \begin{minipage}[t]{0.45\textwidth}
    	\centering
        {\Large Valentin Kocijancic \\ \texttt{\href{mailto:valentin.kocijancic@edu.ece.fr}{valentin.kocijancic@edu.ece.fr}}}
    \end{minipage}   
    \vfill
    \begin{minipage}[t]{0.45\textwidth}
        \centering
        {\Large Hugo Rivière \\ \texttt{\href{mailto:hugo.riviere@edu.ece.fr}{hugo.riviere@edu.ece.fr}}}
    \end{minipage}
    \hfill
    \begin{minipage}[t]{0.45\textwidth}
    	\centering
        {\Large Ethan Petain \\ \texttt{\href{mailto:ethan.petain@edu.ece.fr}{ethan.petain@edu.ece.fr}}}
    \end{minipage}  
    
    
    
    \vfill
    {\large Lyon, le 6 avril 2025\par}
    \vspace{1cm}
    {\normalsize Nous attestons que ce travail est original, qu’il est le fruit d’un travail commun au binôme et qu’il a été rédigé de manière autonome.\par}
\end{titlepage}

\newpage
\tableofcontents
\newpage

\section{Idées en vrac à ne pas oublier}
Down-Sampling $\rightarrow$ C’est pourquoi on applique souvent un filtre passe-bas avant le downsampling, pour supprimer les hautes fréquences qui causeraient de l’aliasing.

\textbf{Down-Sampling :}
Cas typique d’utilisation :
\begin{itemize}
   \item Tu échantillonnes à 32kHz pour avoir une bonne précision,
   \item Tu filtres le signal (par exemple passe-bas à 3kHz),
  \item Puis tu fais un downsampling à 8kHz pour économiser mémoire et CPU.
\end{itemize}

\textbf{Pourquoi échantillonner à 32 kHz ?} \\
C’est lié à une règle de base en traitement du signal : le théorème de Nyquist-Shannon.
\begin{itemize}
\item  "Pour représenter un signal sans perte, il faut échantillonner à au moins 2 fois la fréquence maximale qu'on veut capter." 
\end{itemize}

Donc si on veut capter toutes les fréquences jusqu’à 16 kHz, il faut au moins 32 kHz d’échantillonnage.

Lorsque j'implémente le filtre, avec le bouton poussoir avec le systeme de down sampling, cela me bloque la loop, je pense que puisque je n'ai pas utilisé le buffer circulaire cela me fait utilisé trop de mémoire.

\textbf{Filtre RIF} $\rightarrow$ plus simple à implémenter dans un microcontroleur comme Arduino DUE

Après avoir implémenter le filtre, mes audios sont illisibles sur Audacity, j'ai du réduire le filtre (les coefficient du filtre) pour arriver à reconnaître mon mot. Je pense que je dois laisser un filtre assez fort pour vraiment atténuer de 30 dB lorsque cela dépasse 4 kHz de fréquence. Par ailleurs, même avec un down sampling pour économiser le CPU, un filtrage, et avoir échantillonné l'audio j'arrive quand même à discerner ce que je dis. Je pense que ça sera assez pour appliquer l'algorithm MFCC

Je pense que le choix des coefficients dans le filtre va être pas facile parce que ça à l'air de changer beaucoup les valeurs en sortie. Je dois faire le test pour plusieurs valeurs de coefficient et faire l'algo mfcc et voir ce qui rend le mieux.

J'ai enfin compris la FP4, il faut découper mes valeurs que j'ai en sortie par groupe de 256 valeurs avec une superposition (50\%) et ensuite j'aurai normalement un nombre de frame (64) que je visualise et ensuite que je met dans l'algo MFCC (\url{https://github.com/FouedDrz/arduinoMFCC/blob/main/exemple/MFCCexemple01.ino}) et ensuite je le revisualise et normalement c'est bon.

\section{Objectifs}
Quel est l’objectif de ce document ?
Que va y trouver le lecteur ?
	
\section{Glossaire}
\subsection{Termes}
Renseigner ici sous forme de tableau les principaux termes techniques et leurs définitions.
			
\subsection{Acronymes}
		
Renseigner ici sous forme de tableau les principaux acronymes, leurs signification et leurs explication.
			
\section{L'équipe}
\subsection{Présentation de l’équipe}
			
Qui sont les membres qui composent l’équipe ?
			
Quelles sont leurs compétences et qualités ?
			
\subsection{Organisation de l’équipe}
			
Comment est organisée l’équipe ? Comment est réparti le travail ?

\subsection{Diagramme de Gantt}
Comment est utilisé le temps alloué au projet ?

\section{Contexte et problématique}
\subsection{Contexte}

Quel est le contexte économique et ou sociétal du projet ?

Comment est née l’invention / la technologie du projet, comment a-t-elle évolué ?

\subsection{Problématique}
		
À quelle problématique répond le projet ?
			
\subsection{Spécifications techniques}

Quelles sont les spécifications techniques du projet ?
			
\textbf{NB} : Certains projets d’électronique à l’ECE n’en ont pas.
			
\section{Conception}
\subsection{Architecture fonctionnelle}
		
Quelle est l’architecture fonctionnelle du projet ?
			
\textbf{NB} : Les fonctionnalités sont des verbes à l’infinitif suivi de compléments.
			
À ce stade, aucun choix technique n’est fait.
			
\subsection{Architecture matérielle}
			
Quel matériel est utilisé et pourquoi ?
			
Comment les différentes briques techniques sont connectées entre elles ?
			
\textbf{NB} : Cela peut-être une schématique de circuit électronique.
			
\subsection{Architecture logicielle}
		
Comment fonctionne le programme ?
			
\textbf{NB} : Présenter un algorigramme de votre code si vous en avez-un.
	
\section{Développement}
			
L’idée est de présenter ici comment ont été développés les différents blocs du projet. Cela peut rassembler des calculs théoriques, des choix techniques, etc. et surtout bien expliquer le concept clef derrière	sa fabrication. Le lecteur doit être capable de comprendre les enjeux techniques et de développer le module en question à l’aide de ces sous-sections.
			
\subsection{Module 1}
\subsection{Module 2}
\subsection{Module 3}
		
\section{Tests et validation}
	
Une section au moins aussi importante que celle sur le développement.
		
Il est question ici de montrer les performances techniques du système et de valider le développement module par module puis au global (intégration) en accord avec la partie IV.
		
Chaque résultat (bien souvent des courbes) doit être décrit comme suit :
		
\begin{itemize}
	\item ce qui a été fait ;
	\item ce que l’on est censé obtenir et critère de réussite du test ;
	\item ce que l’on obtient ;
	\item conclusion : validation ou non du bon fonctionnement du module.
\end{itemize}
	
\subsection{Module 1}
\subsection{Module 2}
\subsection{Module 3}
	
\section{Bilan}	
\subsection{État d’avancement}
		
Où en est le projet ? A-t-on atteint les objectifs ?
			
Quels modules restent à finaliser (ou à perfectionner pour être en accord avec les spécifications techniques) ?
			
\subsection{Pertinence de la solution technique}
			
Quelles sont les limites techniques de la solution développée ?
			
Quelles sont les possibilités d’évolution ou de poursuite ?
			
\subsection{Bilan sur le travail d’équipe}
		
Qu’avez-vous appris individuellement ? Quelles compétences vont pouvoir être mises en avant lors de votre prochaine recherche de stage ?
			
Comment l’équipe aurait pu mieux s’organiser ? Proposer un plan d’action pour le prochain projet.
	
\newpage	
\section{Sources}
Documents utilisés et sites internet consultés pour développer le projet.
\textbf{NB} : voir le document « comment rédiger un rapport » sur la page Moodle La Toolbox pour la syntaxe	à utiliser pour vos citations.

Site pour implémenter un filtre FIS : \color{blue}{\url{https://www.labunix.uqam.ca/~boukadoum_m/MIC4220/Notes/FIR.pdf}}\color{black}

Github de Foued Derraz sur l'algorithme MFCC sur Arduino DUE : \color{blue}\url{https://github.com/FouedDrz/arduinoMFCC/tree/main}\color{black}
	
\newpage
\section{Annexes}
Documents volumineux, éventuels codes (\textbf{\textcolor{red}{pas de code dans le rapport}}).



\clearpage

\section*{\LARGE{Annexe}}

\end{document}
